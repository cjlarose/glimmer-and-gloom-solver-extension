\documentclass{article}
\usepackage{amsmath, amssymb, amsthm}

\newtheorem{theorem}{Theorem}
\newtheorem{lemma}[theorem]{Lemma}
\newtheorem{corollary}[theorem]{Corollary}

\title{Reduction of Glimmer and Gloom to Solving a System of Linear Modular Congruences in \(\mathbb{Z}_2\)}
\author{Chris La Rose}

\begin{document}

\maketitle

\section{Introduction}

Glimmer and Gloom is a game played on a finite undirected simple graph \( G = (V, E) \), where each vertex \( v \in V \) is assigned a label \( f_0(v) \in \mathbb{Z}_2 \), with \( 0 \) representing a "dark" vertex and \( 1 \) representing a "light" vertex. A move consists of "clicking" a vertex \( v \), which toggles the state of \( v \) and all its neighbors. The goal is to find a sequence of vertex clicks that results in either all vertices being dark (\( 0 \)) or all vertices being light (\( 1 \)).

\section{Reduction to Linear System}

For each vertex \( v \in V \), define the vector \( c_v \in \mathbb{Z}_2^{|V|} \), where:
\[
    c_v(u) = 
    \begin{cases} 
        1 & \text{if } u = v \text{ or } (u, v) \in E, \\
        0 & \text{otherwise}.
    \end{cases}
\]
The vector \( c_v \) describes the effect of clicking vertex \( v \) on the graph: it toggles the state of \( v \) and all its neighbors.

Let \( A \in \mathbb{Z}_2^{|V| \times |V|} \) be the matrix whose columns are the vectors \( c_v \) for \( v \in V \). A sequence of clicks can be represented by a vector \( x \in \mathbb{Z}_2^{|V|} \), where \( x_i = 1 \) if vertex \( v_i \) is clicked and \( x_i = 0 \) otherwise. Applying this sequence of clicks to the initial labeling \( f_0 \in \mathbb{Z}_2^{|V|} \) results in a new labeling given by:
\[
    A x = f_0 + d \mod 2,
\]
where \( d \in \mathbb{Z}_2^{|V|} \) is the desired final labeling, either the all-dark vector (\( d = 0 \)) or the all-light vector (\( d = 1 \)).

\section{Existence of Solutions}

The equation \( A x = f_0 + d \mod 2 \) is a system of linear equations over \( \mathbb{Z}_2 \). To determine whether a solution exists, we observe that \( A x = f_0 + d \) is consistent if and only if \( f_0 + d \) belongs to the image (column space) of \( A \). 

The matrix \( A \) represents the adjacency structure of the graph \( G \), and its rank determines the number of independent equations in the system. If \( A \) has full rank (i.e., \( \operatorname{rank}(A) = |V| \)), then for each \( f_0 + d \), there exists a unique solution \( x \). If \( A \) does not have full rank, there may be multiple solutions or none, depending on whether \( f_0 + d \) lies in the image of \( A \).

Since each click is an involution (clicking a vertex twice has no effect), the system is guaranteed to have a solution whenever \( f_0 + d \) is in the image of \( A \).

\section{Conclusion}

We have shown that solving the game of Glimmer and Gloom is equivalent to solving the system of linear equations \( A x = f_0 + d \mod 2 \), where \( A \) is the matrix encoding the adjacency structure of the graph, \( f_0 \) is the initial vertex labeling, and \( d \) is the desired final labeling. The existence of a solution depends on the rank of \( A \) and whether \( f_0 + d \) lies in the image of \( A \). If \( A \) has full rank, the system has a unique solution for any \( f_0 \) and \( d \).

\end{document}
