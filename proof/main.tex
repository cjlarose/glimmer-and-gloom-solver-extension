\documentclass{article}
\usepackage{amsmath, amssymb, amsthm}

\newtheorem{theorem}{Theorem}
\newtheorem{lemma}[theorem]{Lemma}
\newtheorem{corollary}[theorem]{Corollary}

\title{Reduction of Glimmer and Gloom to Solving a System of Linear Modular Congruences over \(\mathbb{Z}_2\)}
\author{Chris La Rose}

\begin{document}

\maketitle

\section{Introduction}

Glimmer and Gloom is a game played on a finite undirected simple graph \( G = (V, E) \), where each vertex \( v \in V \) is labeled by a function \( f_0: V \to \mathbb{Z}_2 \), with \( 0 \) representing a ``dark'' vertex and \( 1 \) representing a ``light'' vertex. A player can perform a click operation on any vertex \( v \), which toggles the label of \( v \) as well as the labels of all adjacent vertices. The goal is to find a sequence of click operations that produces a labeling \( d \) such that either all vertices receive a dark label (\(\forall v \in V, d(v) = 0 \)) or all vertices receive a light label (\( \forall v \in V, d(v) = 1 \)).

\section{Reduction to Linear System}

Let \( A \in \mathbb{Z}_2^{|V| \times |V|} \) be the matrix such that

\[
  A_{i,j} =
  \begin{cases}
        1 & \text{if } i = j \text{ or } (v_i, v_j) \in E, \\
        0 & \text{otherwise}
  \end{cases}
\]

That is, \( A \) is the adjacency matrix of the graph \( G \) plus the identity matrix \( I_{|V|} \). The column vectors of this matrix represent the effect on the labels of the vertices when the corresponding vertex is clicked.

Let  \( x \in \mathbb{Z}_2^{|V|} \) be an indicator vector for a set of vertices \( S \subseteq V \), where \( x_i = 1 \) if \( v_i \in S \) and \( x_i = 0 \) otherwise. Clicking on each of the vertices in \( S \) on the graph with initial labeling \( f_0 \in \mathbb{Z}_2^{|V|} \) results in a new labeling given by:
\[
    A x = f_0 + d,
\]
where \( d \in \mathbb{Z}_2^{|V|} \) is the desired final labeling, either the all-dark vector (\( d = \vec{0} \)) or the all-light vector (\( d = \vec{1} \)).

\section{Existence of Solutions}

The equation \( A x = f_0 + d \) is a system of linear equations over \( \mathbb{Z}_2 \). To determine whether a solution exists, we observe that \( A x = f_0 + d \) is consistent if and only if \( f_0 + d \) belongs to the image (column space) of \( A \). 

The matrix \( A \) represents the adjacency structure of the graph \( G \), and its rank determines the number of independent equations in the system. If \( A \) has full rank (i.e., \( \operatorname{rank}(A) = |V| \)), then for each \( f_0 + d \), there exists a unique solution \( x \). If \( A \) does not have full rank, there may be multiple solutions or none, depending on whether \( f_0 + d \) lies in the image of \( A \).

Since each click is an involution (clicking a vertex twice has no effect), the system is guaranteed to have a solution whenever \( f_0 + d \) is in the image of \( A \).

\section{Conclusion}

We have shown that solving the game of Glimmer and Gloom is equivalent to solving the system of linear equations \( A x = f_0 + d \) over \( \mathbb{Z}_2 \), where \( A \) is the matrix encoding the adjacency structure of the graph, \( f_0 \) is the initial vertex labeling, and \( d \) is the desired final labeling. The existence of a solution depends on the rank of \( A \) and whether \( f_0 + d \) lies in the image of \( A \). If \( A \) has full rank, the system has a unique solution for any \( f_0 \) and \( d \).

\end{document}
